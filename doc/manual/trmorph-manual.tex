\documentclass[twocolumn]{article}
\usepackage{etex}\reserveinserts{28}
\usepackage{listings}
\usepackage{ifxetex}
\ifxetex
 \usepackage{fontspec}
 \defaultfontfeatures{Ligatures=TeX}
 \setmainfont{Times New Roman}
\else
 \usepackage{times}
 \usepackage[utf8]{inputenc}
\fi

\usepackage{listings}
\lstset{basicstyle={\color{blue!60!black!90}\tt},%
        columns=fixed,%
        showspaces=false,%
        showstringspaces=false%
        basewidth={0.45em},
        fontadjust=true,
}


\usepackage{graphicx}
\usepackage{multirow}
\usepackage{booktabs}
\usepackage{colortbl}
%\usepackage{setspace}
\usepackage[citestyle=authoryear,dashed=false,bibstyle=authoryear,maxcitenames=2,maxbibnames=6,minnames=1,backend=biber]{biblatex}
\addbibresource{trmorph-manual.bib}


\usepackage{lingmacros}
\usepackage{amsmath}
\usepackage{trmorph-manual}

%\usepackage{polyglossia}
%\setlength{\parskip}{1.2ex}
%\setlength{\parindent}{0em}




\title{TRmorph: A morphological analyzer for Turkish}
\author{Çağrı Çöltekin}
\date{\textbf{Draft:} \today}

\begin{document}
\twocolumn[
\begin{@twocolumnfalse}
\maketitle
\textcolor{red}{%
\textbf{A word of warning:} This document describes the new/development version
of TRmorph. As such, there may be some mismatches between what is
documented here and how the analyzer behaves. This version is a
complete overwrite of the previous version reported in
\cite{coltekin2010}. If you are using the older version (you
shouldn't), this document is probably useless for you.
}
\vspace{1cm}
\end{@twocolumnfalse}
]

\pdfbookmark[0]{TRmorph}{TRmorph}

\section{Introduction}

TRmorph is an open-source\footnote{Current version of TRmorph is
licensed under \href{http://www.gnu.org/licenses/lgpl.html}{GNU Lesser
General Public License}. See README file in the TRmorph distribution
for more information.} finite-state morphological analyzer for
Turkish. The official web site of the analyzer is
\url{http://www.let.rug.nl/coltekin/trmorph}. Since it's release and
introduction in \cite{coltekin2010}, it has been used by many
researchers and NLP practitioners. This document gives a brief
description of how to use the tools that comes with this package, as
well as some implementation details that may be helpful for people
customizing this open-source tool for their own needs.

Earlier versions of trmorph was implemented using SFST
\parencite{schmid2005}, current version is implemented  more
wide-spread finite state description languages \emph{lexc} and
\emph{xfst} from Xerox \parencite{beesley2003}, using Foma
\parencite{hulden2009} as the main development tool. 

The lexc/xfst implementation of TRmorph should compile with any
lexc/xfst compiler without much additional effort.  The only
foma-specific notation used in the morphology description is about
handling duplication, which can also be handled with twolc rules, or
compile-replace \parencite{beesley2003}.\footnote{Compilation with
HFST tools \parencite{linden2009} without modification is not a
problem since HFST uses foma as back end for parsing xfst files.}

\section{How to use it}

\subsection{Compilation from the source}

To compile TRmorph from sources, besides a lexc/xfst compiler such as
foma, you will need a C preprocessor, GNU make on a UNIX-like system
for compiling TRmorph.

If you meet all requirements to build analyzer/generator FST, you
should type \lstinline{make} in the main TRmorph distribution to get
the compiled automaton file \lstinline{trmorph.fst}.

Trmorph also comes with a set of other finite-state tools that can be
useful in various tasks. Currently, tolls for the following is
distributed together with TRmorph.

\begin{itemize}
\item stemming/lemmatization
\item morphological segmentation
\item hyphenation
\item guessing unknown words
\end{itemize}

To compile these tools, you should type specify the FST you want to
build as an argument to \lstinline{make}, e.g., 
\lstinline{make stemmer} 
will build an binary automaton called \lstinline{stem.fst}.
These additional tools are described in Section~\ref{sec:other-tools}.

\subsection{Customizing TRmorph}

TRmorph is an open source utility. As a result, you are free to modify
the source according to your needs. Source code includes some useful
comments on what/how/where things are done. Furthermore, TRmorph can
be customized for some common choices during the compilation. These
options are typically related to more relaxed analysis. For example
whether to allow non-capitalized proper names, or analyze (and
generate) text written in all capitals, or set the decimal and
thousand separator in numbers. These options are set in file
\lstinline{options.h}. The file contains necessary documentation, and
new options are being added based on user requests.

\subsection{Trying it out}

Assuming you build the analyzer \lstinline{trmorph.fst} using foma,
you can simply start foma and use xfst commands implemented in foma to
analyze and generate the words. Here is an example session:

\begin{lstlisting}[basicstyle={\color{blue!60!black!90}\small\tt}]
$ foma
foma[0]: regex @"trmorph.fst";
2.1 MB. 62236 states, 135237 arcs, Cyclic.
foma[1]: up okudum
oku<V><past><1s>
foma[1]: down oku<V><past><1s>
okudum
\end{lstlisting}%stopzone 

The first line is from shell starting foma, the command to foma on the
second line reads file \lstinline{trmorph.fst}. The fourth line asks
for analysis of the verb \xmplt{okudum}{I read-PAST} and the fifth
line is the output of the analysis. The sixth line asks for the
generation of the analysis string obtained, and not surprisingly the
result is the same.

Note that some of the output is suppressed for readability, and this
is one of the rare cases where the analysis is unambiguous. Turkish
morphological analysis is an ambiguous process, and TRmorph does not
try to avoid it during the analysis.\footnote{For most purposes, the
output of the morphological analyzer has little use without
disambiguation. There are quite a few disambiguators reported in the
literature, but, as yet, there are no disambiguators that work with
TRmorph output.}

Once you are convinced that the output may be useful for your
purposes, you will probably want to use it analyzing large amount of
text. For batch analysis tasks  foma's \lstinline{flookup} utility may
fit better.

To use the analyzer with HFST, you need to compile the source
automaton with HFST tools. 

\section{The tagset}

Since the first SFST version, the tagset of TRmorph evolved
considerably, mostly as response to user requests. The description of
the morphology mostly follows  \cite{goksel2005}. However, there are
frequent divergences. This section describes the tags used in TRmorph.
The aim of this section is to help users understand the output of the
system. Occasional discussion of the morphological process is also
included, but this section documents neither the morphology of the
language nor the way it is implemented in TRmorph. The index at the
end of the document also allows easy access to the points where a
particular tag is defined or mentioned in this document.

A clarification of the notation for the surface forms is in order
before starting the documentation of the tagset and related suffixes.
Suffixes in Turkish typically contain under-specified vowels and
consonants that are resolved according to morphophonological rules,
like vowel harmony. These vowels and consonants are indicated with
capital letters listed below.

\begin{description}
\item[A] is realized as either `a' or  `e'.
\item[I] is realized as either `ı', `i', `u' or  `ü'.
\item[D] is realized as either `d' or  `t'.
\item[P] is realized as either `p' or  `b'.
\item[K] is realized as either `k', 'ğ' or  `y'.
\item[C] is realized as either `c' or  `ç'.
\end{description}

A letter in parentheses indicate a buffer consonant or vowel, that may
be dropped in certain contexts.

\subsection{Part-of-speech tags}

All words at least one POS tag. A word may get multiple POS tags, and
even the same tag repetitively. This tracks the changes to POS during
suffixation. This process follows the idea of \emph{inflectional
groups} \parencite[IGs,][]{oflazer1999,oflazer2003}, that are
typically used in Turkish NLP literature. The idea is that a word in
Turkish may start with a particular part of speech due to lexical
properties of the root, may get some inflectional suffixes that modify
this POS, then may get a derivational suffix which changes its POS,
and may again take additional inflections. This process may continue
theoretically indefinitely. Logically, the POS tag of the final word
is the last POS tag in the sequence. However, the other intermediate
forms are also important, and, at least in typical dependency parsing,
can participate in dependency relations with other words by
themselves. For example the analysis of the word
\xmpl{evdekilerinki}\footnote{Rough translation: `the \emph{ones} that belong
to the \emph{ones} in the house', as in `the \emph{books} that belong
to the \emph{people} in the house'.} in example~\ref{ig-example}.

\enumsentence{\label{ig-example}ev\mtag{N}\mtag{loc}\mtag{ki}\mtag{Adj}\mtag{0}\mtag{N}\mtag{pl}\mtag{gen}\mtag{ki}\mtag{Adj}}

The example analysis in (\ref{ig-example}) includes the following IGs

\begin{enumerate}
\item The initial noun `ev' with locative maker.
\item Addition of the suffix \sffx{ki} makes an adjective.
\item The adjective becomes a (pro)nominal with a zero derivation, which takes
plural suffix and genitive marker.
\item Yet another \sffx{ki} is suffixed, and the word becomes an
adjective again.\footnote{More likely reading of this example
includes another zero derivation causing final POS to be again noun.}
\end{enumerate}

TRmorph, by default, does not mark IG boundaries explicitly. However,
one can easily trace the IG changes following the POS tags. All POS
tag names start with a capital letter, while other tags always start
with a lowercase letter or number.

All part-of speech tags used in TRmorph are listed in
Table~\ref{tbl:pos-tags}.

\begin{table}[t]
\caption{\label{tbl:pos-tags}The list of part of speech tags in
TRmorph along with the corresponding tags in some of the output
produced by in \citeauthor{oflazer1994}'s (\citeyear{oflazer1994})
morphological analyzer.}
\begin{center}
\begin{tabular}{lll}\toprule
Tag          & Description       & Oflazer's tag \\
\toprule
\mtagD{Alpha} & Symbols of alphabet & ?\\
\mtagD{Adj}   & Adjective         & +Adj\\
\mtagD{Adv}   & Adverb            & +Adverb\\
\mtagD{Cnj}   & Conjunction       & +Conjunction\\
\mtagD{Det}   & Determiner        & +Det\\
\mtagD{Exist} & \xmpl{var} and \xmpl{yok}    & +Adj\\
\mtagD{Ij}    & Interjection      & +Interj \\
\mtagD{N}     & Noun              & +Noun\\
\mtagD{Not}   & \xmpl{değil}      & +Verb\\
\mtagD{Num}   & Number            & +Num\\
\mtagD{Onom}  & Onomatopoeia      & ?\\
\mtagD{Postp} & Postposition      & +Postp\\
\mtagD{Prn}   & Pronoun           & +Pron\\
\mtagD{Punc}  & Punctuation       & +Punc\\
\mtagD{Q}     & Question particle & +Ques, also +QuesP\\
\mtagD{V}     & Verb              & +Verb\\
\bottomrule
\end{tabular}
\end{center}
\end{table}

Most POS tags are self explanatory, and does not require much
explanation. The following part of speech tags are somewhat unusual
and deserves some explanation.

\begin{description}
\item[\mtag{Exist}] is used for two words
\xmplt{var}{existent/present} and  \xmplt{yok}{non-existent/absent},
where the latter is marked as \mtag{neg}.
These words behave mostly like nouns in their predicate function (with
zero copula), but marking them simply as nouns does blur their
function.  
\item[\mtag{Not}] is used for \xmplt{değil}{not} only. Like \xmpl{var}
and \xmpl{yok}, \xmpl{değil} also behaves like nominal predicates. But
again, marking it as noun or verb hides the fact that it has a special
function.
\item[\mtag{Q}] is used for the question particle \sffx{mI}. The
question particle is written separately from the predicate it
modifies. However, the preferred analysis of question particle 
in TRmorph is together with the predicate. This ensures that 
it follows a correct form of the predicate and vowel harmony is
applied correctly. However, since we do not assume that the input is
tokenized with this assumption, this form make sure that the input is
analyzed.

This tag is related to the lowercase \mtag{q} tag, which is used if
input is tokenized such that the predicate and the question particle
form a single token. The question words corresponding to wh-words in
English are marked with the corresponding POS tag together with the
tag \mtag{qst}.
\end{description}

\subsection{Subcategorization of lexemes}

Besides the major major POS tags or word classes discussed above, we
often we assign subcategories to a lexeme. Typically the
subcategorization is applied to a root form in the lexicon, but some
morphemes and POS tags after a derivation may also receive a
subcategory. Subcategories defined here are features of a morpheme
that do not have a surface realization. Representing these using a
different notation allows to make this distinction, and a
surface--analysis mapping that is (almost) one-to-one.  However, these
tags can be viewed as any other feature of a word (or IG), and the
notation can be replaced with more uniform notation easily.

The subcategories generally mark semantic differences, but they may
also result in morphosyntactic differences. Lexical subcategorization
in TRmorph output is marked using the syntax
\formattag{Cat:subcat$_1$:subcat$_2$:\ldots}, where `Cat' is a major
category and `subcat$_1$', `subcat$_2$' and so on are sub categories.
A typical example of a subcategory is the \emph{proper} nouns, which
are tagged as \mtag{N:prop}. 

The following lists subcategories in used in TRmorph for all POS tags
that may receive a subcategory.

\begin{description}
\item[Nouns] besides the tag \mtag[def]{N:prop} marking proper names, 
abbreviated nouns are marked with the tag \mtag[def]{N:abbr}. If the 
lexeme is an abbreviated proper name, then the tag would be 
\mtag[def]{N:prop:abbr}. 

\item[Conjunctions] are subcategorized as
\emph{coordinating}, \emph{adverbial} or \emph{subordinating} 
conjunctions, marked using tags \mtag[def]{Cnj:coo}, \mtag[def]{Cnj:adv},
\mtag[def]{Cnj:sub} respectively. 

The last one of these categories, \mtag{Cnj:sub}, include only a
limited set of conjunctions which come first in a subordinate clause.
These words currently are \xmpl{ki}, \xmpl{eğer} and \xmpl{şayet}, all
borrowings from Persian. The other subordinating particles/words occur
at the end of the subordinate clause, and they are marked as
postpositions (\mtag{Postp}) described below.  Furthermore, most
of the subordination is in Turkish is done through suffixation which
is described in Section~\ref{ssec:subordination}.

\item[Pronouns] Pronouns are further categorized as \emph{personal},
\emph{demonstrative} and \emph{locative} pronouns, marked using
\mtag[def]{Prn:pers}, \mtag[def]{Prn:dem}, \mtag[def]{Prn:locp} 
respectively. Furthermore, the pronouns that question that form
questions, like \xmplt{kim}{who}, and \xmplt{ne}{what}, are
marked as \mtag[def]{Prn:qst}. Like in conjunctions both subcategory
markers can be present. For example \xmplt{kim}{who} would be marked
as \mtag{Prn:pers:qst}.

Besides the above subcategories, pronouns get number and person
agreement markers. These markers can be useful in subject-predicate
agreement as well as some other constructions (such as
genitive-possessive construction involving pronouns). However, the
agreement in Turkish is far from being trivial (see
\cite[pp.116--122]{goksel2005}). The markers 
\mtag[def]{Prn:pers:1s},
\mtag[def]{Prn:pers:2s},
\mtag[def]{Prn:pers:3s},
\mtag[def]{Prn:pers:1p},
\mtag[def]{Prn:pers:2p} and
\mtag[def]{Prn:pers:3p} are tags used for the person and number the
pronominal will agree. The reflexive pronoun \xmpl{kendi(si)} is
marked as \mtag[def]{Prn:pers:refl}

Subcategorization of pronouns, particularly as personal pronouns, are
sometimes not a clear decision. Subcategories of pronouns are left
unspecified even though they are primarily used as personal pronouns,
and some pronoun marked as personal pronouns may refer to entities
other than people.

\item[Determiners] are marked based on their definiteness. 
\emph{Demonstrative} determiners are marked \mtag[def]{Det:def} and
indefinite determiners are marked \mtag[def]{Det:indef}. The question
words that take place of determiners \xmplt{ne kadar}{how much} and
\xmplt{hangi}{which} are tagged with \mtag{Det:qst}. 

Further subcategorization of determiners (for example 
quantifiers) can be implemented in the future.

\item[Postpositions] are always subcategorized in two dimensions.
First subcategory is the category of the resulting postpositional
phrase, either an \emph{adjectival} or \emph{adverbial} phrase, marked
as \mtag[def]{Postp:adj} and \mtag[def]{Postp:adv} respectively.

Postpositions choose the noun phrase complements.
Besides the category of the resulting phrase, postpositions also
include a tag specifying the requirement for the complement noun
phrase. The complement requirement tag is formed a concise description of the
requirement followed by capital letter `C'. The postpositions that require 
the complement to be in \emph{ablative}, \emph{dative} and
\emph{instrumental} cases are marked 
\mtag[def]{Postp:ablC},
\mtag[def]{Postp:datC},
and \mtag[def]{Postp:insC} respectively. Postpositions that require
non-case marked complement receives \mtag[def]{Postp:nomC}. The
postpositions that require the noun phrase to be suffixes with either
\sffx{lI} or \sffx{sIz} are marked with
\mtag[def]{Postp:liC}.\footnote{Like \sffx{(y)lA} that we mark as
\mtag{ins}, these suffixes are typically considered derivational suffixes, 
however their use resemble case markers.}
Finally, postpositions that require a number as their complement are
marked with \mtag[def]{Postp:numC}

\item[Numbers] are tagged as \mtag[def]{Num:ara} for Arabic numerals,
and \mtag[def]{Num:rom} for Roman numerals. Numbers that are spelled
out are not marked with a subcategory marker (but still marked as
\mtag{Num}). Besides numbers the question word \xmplt{kaç}{how many}
is also tagged as a number with a sub tag specifying that it is a
question word, resulting in \mtag[def]{Num:qst}.

\item[Verbs] are currently not subcategorized in TRmorph. 

Subcategorizing verbs \emph{transitive} and \emph{intransitive}, or
marking all types (cases) of noun phrase complements a verb can take
is planned and some early steps are underway as of this writing (July
2013).
\item[Adverbs] are not currently subcategorized, except a few adverbial
question words for which the tag \mtag[def]{Adv:qst} is used.

% \item[Interjections] The question word \xmplt{hani}{where} is marked
% as an interjection as well as being marked as a \mtag{Prn:qst:loc},
% resulting in tag \mtag[def]{Ij:qst}. 
\item[Exist] The tag \mtag[def]{Exist} exists only for two words
\xmplt{var}{existent/present} and  \xmplt{yok}{non-existent/absent}. 
Since \xmpl{yok} is the negative of \xmpl{var}, it is tagged as
negative: \mtag[def]{Exist:neg}.
\end{description}

Some verbs, nouns, adjectives, adverbs and conjunctions are formed
from more than one written words. Some of these are local, like the
adverb \xmplt{apar topar}{hurriedly}, but some may be split like the
conjunction \xmpl{ya} \xmplt{ya evdedir ya iş yerinde}{s/he is either
at home or the office}. Furthermore, some of individual `words' in
such constructions cannot be used by themselves, like \xmpl{topar}
above. If the non-split multi-word expressions are input to the
analyzer together, they are analyzes like other words of the same
class. However, if they are input word-by-word a sub tag
\formattag{:partial} is added to the main POS tag. For example
\xmpl{apar} and \xmpl{topar} is be tagged as \mtag[def]{Adv:partial}
and \xmpl{ya} is tagged as \mtag[def]{Cnj:partial} (more precisely
\mtag[def]{Cnj:coo:partial}). Similarly, currently the tags 
\mtag[def]{N:partial}, 
\mtag[def]{Adj:partial} and
\mtag[def]{V:partial} are used for nouns, adjectives and verbs
respectively.

\subsection{Nominal morphology and noun inflections}

Nouns, pronouns, adjectives and adverbs in Turkish are considered
nominals. Most adverbs and adjectives can function as nouns (or
pronouns). For example \xmplt{mavi}{blue} may mean `the blue one', or
\xmplt{hızlı}{fast} may mean `the fast one'. In TRmorph this is
handled by allowing any adjective or adverb to become an noun with a
\emph{zero derivation}. A zero derivation is always marked with tag
\mtag{0} and followed by the new POS tag, in this case \mtag{N}. 

Nouns can receive plural suffix, one of the possessive suffixes and
one of the case suffixes. All of these inflections are optional.
However, when they co-occur, they have to occur in this order. Full
list of noun inflections are presented in
Table~\ref{tbl:noun-inflections}.

\begin{figure}
\resizebox{\linewidth}{!}{%
\tikzinput{figures/nomfsa1}
}
\caption{\label{fig:noun-inflections}Automata depicting noun
inflections. The edge CASE1 represents the locative and 
ablative suffixes, CASE2 represents all other case-like suffixes. The
reason for for differentiation is due to the fact that the state
CASE1 can be followed by \sffx{ki}.}
\end{figure}

\begin{table}[t]
\caption{\label{tbl:noun-inflections} Noun inflections. }
\begin{center}
\begin{tabular}{llll}\toprule
\multicolumn{2}{l}{\textbf{Function}} & \textbf{surface} & \textbf{tag} \\
\toprule
&Plural  & \sffx{lAr} &  \mtag[def]{pl} \\
\midrule
\multirow{6}{*}{\rotatebox{90}{Possessive}}
&First person singular  & \sffx{(I)m} &  \mtag[def]{p1s} \\
&Second person singular  & \sffx{(I)n} &  \mtag[def]{p2s} \\
&Third person singular  & \sffx{(s)I} &  \mtag[def]{p3s} \\
&First person plural  & \sffx{(I)mIz} &  \mtag[def]{p1p} \\
&Second person plural  & \sffx{(I)nIz} &  \mtag[def]{p2p} \\
&Third person plural  & \sffx{lArI} &  \mtag[def]{p3p} \\
\midrule
\multirow{6}{*}{\rotatebox{90}{Case}}
&Accusative               & \sffx{(y)I}  & \mtag[def]{acc}\\
&Dative                   & \sffx{(y)A}  & \mtag[def]{dat}\\
&Ablative                 & \sffx{DAn}   & \mtag[def]{abl}\\
&Locative                 & \sffx{DA}    & \mtag[def]{loc}\\
&Genitive                 & \sffx{(n)In} & \mtag[def]{gen}\\
&Instrumental/commutative & \sffx{(y)lA} & \mtag[def]{ins}\\
\bottomrule
\end{tabular}
\end{center}
\end{table}

If there is a plural marker analysis symbol after the \mtag{N} will
include a \mtag{pl}. TRmorph does not mark for singular. If a noun is
not marked for plural, it is assumed to be singular. 

%There are a small number of loan words that are plural in their
%language of origin, e.g., \xmplt{eşya}{belongings}. ...

Possessive markers follow either nominal stem, or the plural marker.
Besides marking for possession, these suffixes derive pronouns when
attached to adjectivals(e.g., determiners or adjectives).

The first five cases in Table~\ref{tbl:noun-inflections} are commonly
recognized cases in Turkish. The \emph{instrumental/commutative}
marker also behaves like case suffixes. There are two more suffixes
\sffx{lI} and \sffx{sIz} (resembling to abessive case in its
function), that can occupy the same slot, which are marked with tags
\mtag[def]{li} and \mtag[def]{siz} respectively.

The \sffx{(s)I} suffix, listed as \mtag{p3s} in
Table~\ref{tbl:noun-inflections}, is highly ambiguous. One of its many
functions that can be confused with the possessive suffix is forming
noun compounds. TRmorph marks this usage of the suffix \sffx{(s)I} as
\mtag[def]{ncomp}. In this use, this suffix always causes ambiguities.
Besides the fact that a noun suffixed with \sffx{(s)I} can be 
either marked for possession or head of a noun compound, since
one of the two \sffx{(s)I} suffixes following each other is deleted
from the surface, it can be both (a noun compound marked for
possession).

The case (or case-like) suffixes change the role of the noun (or the
noun phrase headed by the noun) in the sentence. For example a
locative marked noun phrase may function as an adverb (\xmpl{saat
dokuz\emph{da} görüşurüz}) or an adjective (\xmpl{yedi yaşında
çocuk}). However, following the common practice in the literature we
do not attempt to mark possible POS changes after case-like markers.

\subsection{The suffix \sffx{ki}}

The suffix \sffx{ki}, tagged as \mtag[def]{ki}, attaches to a locative
or ablative marked nouns.  The resulting word functions as an
adjective or pronoun. In both cases, TRmorph marks the transition to
adjective. For example, \xmpl{evdeki} is analyzed as
`ev\mtag{N}\mtag{loc}\mtag{ki}\mtag{Adj}'. Since all adjectives are
allowed to become a noun through a zero derivation, the pronoun
reading is intended to be represented by this change. For example, the
intended analysis for \xmplt{evdeki kitap}{the book in the house} is
`ev\mtag[noindex]{N}\mtag[noindex]{loc}\mtag[noindex]{ki}\mtag[noindex]{Adj}',
while analysis for \xmplt{evdeki uyuyor}{the one/person in the house is
sleeping} suffixes `\mtag{0}\mtag{N}' to this analysis.

The (pro)noun formed by \sffx{ki} can further be suffixed with other
nominal suffixes. Although the number of iterations using multiple
\sffx{ki} suffixes are limited in practice, there is no principle
limit, causing in-principle unbounded length for a word in Turkish.

\subsection{\label{ssec:nompred}Tags related to nominal predicates}

Any nominal in Turkish may become a predicate by attaching one of the
copular suffixes \sffx{(y)DI}, \sffx{(y)mIş}, \sffx{(y)sA} or
\sffx{(y)}. These suffixes correspond to \emph{past},
\emph{evidential}, \emph{conditional}, and \emph{present} predicates
involving the copula `be'. The copular markers has to follow one of
the verbal person agreement markers. For example
\xmplt{öğrenciy\emph{di}k}{We were students}),
\xmplt{öğrenciy\emph{miş}ler}{They were (evidentially) students}),
\xmplt{öğrenciy\emph{se}n}{If you are/were a student},
\xmplt{öğrenci\emph{y}im}{I’m a student}). Since the third person
singular agreement suffix has no surface form and the buffer
\sffx{(y)-} does not surface in this case, any nominal without
additional copular or person suffixes serve as a nominal predicate
with present copula and third person singular agreement. Additionally,
since a predicate with third person singular agreement also agrees
with a third person plural agreement, we additionally mark such a noun
as having present copula and third person plural agreement (for
example, \xmplt{babam öğretmen, annem ve ablam doktor}{my father is a
teacher, my mother and older sister are doctors}). 

TRmorph handles this process by allowing any noun to first became a
verb by a zero derivation, and then marking them verb with the
appropriate copula and the person agreement marker. The tags for
copula are \mtag[def]{cpl:pres}, \mtag[def]{cpl:past},
\mtag[def]{cpl:evid} and \mtag[def]{cpl:cond} for , \emph{present},
\emph{past}, \emph{evidential} and \emph{conditional} copula
respectively. Last three tags are also possible after a verb with a
tense/aspect/modality suffix, and will be discussed further below.
Example analyses (without root) for the examples above would be as
follows:\\
\begin{tabular}{ll}
\xmpl{öğrenciydik}&\mtag{N}\mtag{0}\mtag{V}\mtag{cpl:past}\mtag{1p}\\
\xmpl{öğrenciymişler}&\mtag{N}\mtag{0}\mtag{V}\mtag{cpl:evid}\mtag{3p}\\
\xmpl{öğrenciysen}&\mtag{N}\mtag{0}\mtag{V}\mtag{cpl:cond}\mtag{2s}\\
\xmpl{öğrenciyim}&\mtag{N}\mtag{0}\mtag{V}\mtag{cpl:pres}\mtag{1s}\\
\xmpl{öğretmen}&\mtag{N}\mtag{0}\mtag{V}\mtag{cpl:pres}\mtag{3s}\\
\xmpl{doktor}&\mtag{N}\mtag{0}\mtag{V}\mtag{cpl:pres}\mtag{3p}\\
\end{tabular}


Besides copular suffixes, the suffix \sffx{(y)ken} (making adverbials
from verbs, discussed below) may occupy the same slot as the copular
suffixes.

The nominal predicate with a copula and person agreement may be
followed by marker \cite{goksel2005} call `generalizing modality
marker', the suffix \sffx{DIr}. It is particularly common with
\mtag[noindex]{3s} case. The tag for this marker in TRmorph is
\mtag[def]{dir}.

\subsection{Number inflections}

The suffix \sffx{(ş)Ar}, tagged \mtag[def]{dist}, attached to numbers
form \emph{distributive} numerals. Besides the numbers (written as
numerals or spelled out), question word \xmplt{kaç}{how many} may also
get this suffix, and tagged with \mtag{dist}.

The ordinal numerals are formed using the suffix \sffx{-(I)ncI}, and
tagged as \mtag[def]{ord}.

Percent sign before numerals are treated like a prefix, and tagged as
\mtag[def]{perc}.

\subsection{Verbal voice suffixes}

Turkish verbs can be marked as \emph{reflexive}, \emph{reciprocal},
\emph{causative} and \emph{passive} voice. The tags used for these
functions are 
\mtag[def]{rfl},
\mtag[def]{rcp},
\mtag[def]{caus} and
\mtag[def]{pass}, respectively. The first two are rather unproductive
while causative and passive forms are productive. Furthermore,
causative suffix can be used repetitively.\footnote{Again, although this is
limited in practice there is no principled limit on how many causative
suffixes one can string on after another.} With some verbs, double
causative suffix has the same semantics as single suffix. TRmorph does
not treat these cases separately. If surface string has double
causative suffixes, the analysis will include two \mtag{caus} tags,
regardless of its semantics.

Despite the fact that most grammar books list voice suffixes under
inflectional morphology, TRmorph treats them as derivations, i.e., a
\mtag{V} tag follows the voice related tags.

\subsection{Compound verbs}

A verbal stem (possibly including voice suffixes) may be followed by a
set of suffixes which are forms of other verbs. These suffixes are
listed in Table~\ref{tbl:compound-verb}.

\begin{table}[t]
\caption{\label{tbl:compound-verb}Suffixes that make compound verbs.}
\begin{center}
\begin{tabular}{lll}\toprule
Suffix        & Tag             & Expresses\\
\toprule
\sffx{(y)Abil}& \mtag[def]{abil}& ability\\
\sffx{(y)Iver}& \mtag[def]{iver}& immediacy\\
\sffx{(y)Agel}& \mtag[def]{agel}& habitual/long term\\
\sffx{(y)Adur}& \mtag[def]{adur}& repetition/continuity\\
\sffx{(y)Ayaz}& \mtag[def]{ayaz}& almost \\
\sffx{(y)Akal}& \mtag[def]{akal}& stop/freeze in action\\
\sffx{(y)Agör}& \mtag[def]{agor}& somewhat like \mtag{iver}\\ % but used with conditional
%\sffx{(y)Akoy}& \mtag[def]{agor}& --\\
\bottomrule
\end{tabular}
\end{center}
\end{table}

The first three suffixes in this Table~\ref{tbl:compound-verb} are
relatively productive, the others are rare or their use are
lexicalized. Although rare, more than one these suffixes may attach to
the same stem. 

The form of \mtag{abil} in a negative verb is \sffx{(y)A}, and unlike
the rest of the suffixes listed in Table~\ref{tbl:compound-verb} it
follows the negative marker. 

Currently, these suffixes are not marked as derivations (causing an IG
change).

\subsection{Negative marker}

Negation of a verbal predicate is indicated with the suffix \sffx{mA},
and marked simply as \mtag[def]{neg}. For nominal predicates do not
get this suffix, instead the particle \xmpl{değil} is used.

\subsection{Tense/aspect/modality markers}

The verbal suffixes and tags described so far forms a non-finite verb.
A verb with a set of suffixes described above either becomes a finite
word by taking one of the tense, aspect and modality (TAM) markers or
can be subject to subordination and becomes nominalized.

The list of TAM suffixes, the corresponding tags and brief
descriptions are given in Table~\ref{tbl:tam}. 

\begin{table}[t]
\caption{\label{tbl:tam} Tense/aspect/modality markers. The usage of
suffix \sffx{(y)A} to express conditional aspect is informal, and
rather restrictive. Aorist suffix is highly irregular. The choice of
\sffx{Ar} and \sffx{Ir} depends on the stem. The \sffx{z} form occurs
only after negative marker, and it is not realized on the surface if
it precedes first person agreement suffixes.}
\begin{center}
\begin{tabular}{lll}\toprule
Tag & Suffix & Description \\
\toprule
\mtag[def]{evid} &\sffx{mIş}     & evidential past (perfective)\\
\mtag[def]{fut}  &\sffx{(y)AcAk} & future \\
\mtag[def]{obl}  &\sffx{mAlI}    & obligative \\
\mtag[def]{impf} &\sffx{mAktA}   & imperfective \\
\mtag[def]{cont} &\sffx{(I)yor}  & imperfective \\
\mtag[def]{past} &\sffx{DI}      & past (perfective)\\
\mtag[def]{cond} &\sffx{sA},\sffx{(y)A},\sffx{} & conditional \\
\mtag[def]{opt}  &\sffx{(y)A}    & optative \\
\mtag[def]{imp}  &\sffx{}        & imperative \\
\mtag[def]{aor}  &\sffx{Ar},\sffx{Ir},\sffx{z} & aorist \\
\bottomrule
\end{tabular}
\end{center}
\end{table}

\subsection{Person and number agreement}

After TAM markers a finite verb requires one of the person and number
agreement markers. The surface form of the person agreement markers
change depending on the suffixes they follow.
Table~\ref{tbl:agreement} lists the person agreement markers and their
surface form according the TAM of the verb they attach to.

\begin{table}[t]
\caption{\label{tbl:agreement}Verbal person agreement markers. The
first character of the person agreement marker indicates the person,
and second one indicates number (\emph{s}ingular or \emph{p}lural).
The suffixes listed in the column column marked `TAM1' follow the TAM markers
\mtag{evid},\mtag{fut},\mtag{obl},\mtag{impf} and \mtag{cont} as well
as the evidential copula \mtag{cpl:evid} and nominal predicates. 
The same set of suffixes also follow positive verbs with \mtag{aor}
without a negative marker. 
The suffixes on column marked `TAM2' are used after \mtag{past} and
\mtag{cond} as well as the corresponding copular markers
\mtag{cpl:past} and \mtag{cpl:cond}.}
\begin{center}
\begin{tabular}{lllll}\toprule
Tag  & TAM1 & TAM2 & optative & imperative \\
\toprule
\mtag[def]{1s}&
    \sffx{(y)Im}     &\sffx{m}     &\sffx{(y)Im} & *\\
\mtag[def]{2s}&
    \sffx{sIn}       &\sffx{n}     &\sffx{sIn}   &\sffx{} \\
\mtag[def]{3s}&
    \sffx{}          &\sffx{}      &\sffx{}      &\sffx{sIn}\\
\mtag[def]{1p}&
    \sffx{(y)Iz}     &\sffx{K}     &\sffx{lIm}   & *\\
\mtag[def]{2p}&
    \sffx{sInIz}     &\sffx{nIz}  &\sffx{sInIz}  &\sffx{(y)In},\sffx{(y)InIz}\\
\mtag[def]{3p}&
    \sffx{lAr}       &\sffx{lAr}  &\sffx{lAr}    &\sffx{sInlAr},\sffx{}\\
\bottomrule
\end{tabular}
\end{center}
\end{table}

%The use of these agreement markers for the purposes of
%subject--predicate agreement require more than just looking for the
%matching tag.

\subsection{Copular markers and \sffx{DIr}}

The copular suffixes discussed in Section~\ref{ssec:nompred} can also
be attached to a tensed verb, typically forming complex tenses. These
suffixes are \sffx{(y)DI}, \sffx{(y)mIş} and \sffx{(y)sA}, tagged as
\mtag[def]{cpl:past},
\mtag[def]{cpl:evid} and
\mtag[def]{cpl:cond}, respectively.

The conditional copula \sffx{(y)sA} can co-occur with other copular
markers. When there is a copular suffix, person agreement suffixes
normally attach after the first copula. However the third person
plural suffix may be after the TAM marker or second copular suffix as
well.

Similar to the nominal predicates with a copula, copular suffixes may
follow the `generalizing modality marker' \sffx{DIr} tagged as
\mtag[def]{dir}.

\subsection{Question particle}

Question particle \sffx{mI}, tagged as \mtag[def]{Q}, is normally
written separately. However, it has an intimate relationship between
the verb or the nominal predicate it attaches to. First, it typically
is attached to a tensed verb without a person agreement. In which
case, the person agreement and the suffixes that may follow are
attached to the question particle. Second, it follows the vowel
harmony rules, and the underspecified vowel on \sffx{mI} is realized
based on the last vowel of the verb.   As a result the question particle can only be
analyzed (and generated) precisely together with the word it is
attached to.

If tokenized together with the predicate, TRmorph will swallow the
space in between the predicate and the \sffx{mI} and analyze it
altogether. In this case the tag \mtag[def]{q} is used. Furthermore,
it is a common spelling mistakes to write the question particle
together with the related word. TRmorph can instructed to 
to accept this common mistake during compile time, in which case the
tag will again be \mtag{q}.

\subsection{\label{ssec:subordination}Subordination}

A set of suffixes attached to an `untensed' verb, a verb without any
TAM markers, results in the phrase headed by the verb to become a
subordinate clause. TRmorph follows the description in
\cite{goksel2005}, and makes the distinction between three different
forms of subordination. First, a set of suffixes produce \emph{verbal
noun}s from a non-finite verb. The resulting words function as nouns,
and with some limitation they can receive all nominal inflections. The
second group form \emph{participles}, which form relative clauses.
Participles can also take any of the nominal inflections with few
restrictions. The last group, \emph{converbs}, form adverbials and
they are much more restricted in terms of the morphemes attached to
them. The suffixes that for any of these nominal forms overlap
significantly.

TRmorph uses the tag structure \formattag{type:subtype} for marking
subordinating suffixes. The first part, \texttt{type}, is one of
\texttt{vn}, \texttt{part} and \texttt{cv} for verbal nouns,
participles and converbs, respectively. The second, \texttt{subtype}, 
part indicate a further distinction of the function of the suffix,
a relevant linguistic abbreviation, but sometimes a version of the
surface form of the suffix. The tags used for all three types of
subordinating suffixes are listed in Table~\ref{tbl:subord}.

Since verbal nouns participles and converbs derive nominal, adjectival
and adverbial phrases, respectively, POS tags, \mtag{N}, \mtag{Adj}
and \mtag{Adv}, follow these tags.

\begin{table}[t]
\caption{\label{tbl:subord}Subordinating suffixes and tags used for
subordinating suffixes.}
\begin{center}
\begin{tabular}{ll}\toprule
Tag         & Suffix     \\
\toprule
\mtag[def]{vn:inf}      &\sffx{mA}  \\
\mtag[def]{vn:inf}      &\sffx{mAK}    \\
\mtag[def]{vn:yis}      &\sffx{(y)Iş}   \\
\mtag[def]{vn:past}     &\sffx{DIk} \\
\mtag[def]{vn:fut}      &\sffx{(y)AcAk} \\
\mtag[def]{vn:res}      &\sffx{(y)An}   \\
\midrule
\mtag[def]{part:past}   &\sffx{DIk} \\
\mtag[def]{part:fut}    &\sffx{(y)AcAk} \\
\mtag[def]{part:pres}   &\sffx{(y)An}   \\
\midrule
\mtag[def]{cv:ip}       &\sffx{(y)Ip}   \\
\mtag[def]{cv:meksizin} &\sffx{mAksIzIn}    \\
\mtag[def]{cv:ince}     &\sffx{(y)IncA} \\
\mtag[def]{cv:erek}     &\sffx{(y)ArAk} \\
\mtag[def]{cv:eli}      &\sffx{(y)AlI}  \\
\mtag[def]{cv:dikce}    &\sffx{DIkCA}   \\
\mtag[def]{cv:esiye}    &\sffx{(y)AsIyA}    \\
\mtag[def]{cv:den}      &\sffx{dAn} \\
\mtag[def]{cv:den}      &\sffx{zdAn}    \\
\mtag[def]{cv:cesine}   &\sffx{CAsInA}  \\
\mtag[def]{cv:ya}       &\sffx{(y)A}  \\
\mtag[def]{cv:ken}      &\sffx{(y)ken}  \\
\bottomrule
\end{tabular}
\end{center}
\end{table}

Some of the suffixes have multiple functions and may derive more than
one type of subordinated clauses. Furthermore TRmorph will have some
spurious ambiguity because of the fact that any adjective, hence a
word suffixed with an participle, is allowed to become a noun with a
zero derivation.

The list in Table~\ref{tbl:subord} follows \cite{goksel2005}, except
the suffixes listed by them as converbial suffixes that require a
postposition. The postposition in these cases will signal the
adverbial function of postpositional phrase.

Most of these suffixes attach to an untensed verb. Except, the suffix
\sffx{(y)ken} which behaves much like the copular suffixes discussed
above. Furthermore, the \sffx{(y)A} in its subordinating function
is typically used together with reduplication, e.g., \xmplt{koşa
koşa}{run-(y)A run-(y)A = hurriedly}, but also occurs in words like
\xmpl{diye}, where it does not need reduplication.\footnote{One may
also analyze \xmpl{diye} as a postposition, as it's use as
subordinator is semantically unlike the others uses of \sffx{(y)A}.}

Besides, some of the TAM markers, namely \mtag{aor}, \mtag{evid},
\mtag{fut}, \mtag{makta} and less commonly \mtag{cont}, derive
adjectivals resembling relative clauses. TRmorph handles this by
analyzing any verb with one of these TAM markers without further
suffixes (e.g., agreement markers) as an adjective. For example, the
word \xmpl{okunmuş} in \xmplt{okunmuş kitap}{red book = the book that
was read} is analyzed as
`oku\mtag{V}\mtag{pass}\mtag{V}\mtag{evid}\mtag{Adj}'.

\subsection{Productive derivational morphemes}

Almost all the tags and relevant morphological process above are
described as part of inflectional morphology in most grammar books.
The suffixes described here are the ones that are traditionally
considered derivational suffixes. Some of these suffixes, for example
\sffx{lI} and \sffx{sIz} discussed earlier, may attach to word forms
that are already inflected by other suffixes. Others normally attach
only to the stem and produce another stem.

Of these suffixes, the noun--verb derivation suffix \sffx{lA} causes a
large number of ambiguous analyses since it is part of many other
suffixes. These, for example, include the plural suffix \sffx{lAr}
whose remainder \sffx{r} is also match with a verbal suffix (aorist).
Hence, including \sffx{lA} in the analysis causes an increase in the
analyses of any plural noun. Currently, TRmorph analyzes \sffx{lA}
only after onomatopoeia. The rest of the verbs derived from nouns
using this suffix are lexically specified.

Besides these sources of possible erroneous over-analyses, the
derivational morphology specification in TRmorph over-generates in
some cases. In particular, any form of the diminutive suffix is
allowed to attach to any noun. The nouns typically do select for which
suffix to use. However, there is no clear-cut way of predicting this.
As a result, if used for generation, TRmorph will generate all
diminutive suffixes for any noun, most likely only one of them being
correct. 

Furthermore, TRmorph does not limit the number of derivational
suffixes that can be stringed one after another other. In reality this
is a lot more restricted. 

\begin{table}[t]
\caption{\label{tbl:deriv}Derivational morphemes analyzed by TRmorph.
The column `Derivation' lists the POS changes using two letter
codes the first letter is the original POS, and the second one is the
POS after the suffixation. Here, N means noun, J means adjective, A
means adverb, V means verb and O is onomatopoeia.}
\begin{center}
\begin{tabular}{lll}\toprule
Tag              & Suffix        & Derivation \\
\toprule
\mtag[def]{li}   & \sffx{lI}     & NA NJ \\
\mtag[def]{siz}  & \sffx{sIz}    & NA NJ \\
\mtag[def]{lik}  & \sffx{lIk}    & NN JN AN \\
\mtag[def]{dim}  & \sffx{CIk}    & NN \\
                 & \sffx{cAk}    & \\
                 & \sffx{(I)cAk} & \\
                 & \sffx{cAğIz}  & \\
\mtag[def]{ci}   & \sffx{CI}     & NN NJ \\
\mtag[def]{ca}   & \sffx{CA}     & NA AA JJ MJ \\
\mtag[def]{yici} & \sffx{(y)IcI} & VJ \\
\mtag[def]{cil}  & \sffx{CIl}    & NJ \\
\mtag[def]{gil}  & \sffx{gil}    & NN \\
\mtag[def]{lan}  & \sffx{lAn}    & JV \\
\mtag[def]{las}  & \sffx{lAş}    & NV JV \\
\mtag[def]{yis}  & \sffx{yIş}    & VN \\
\mtag[def]{esi}  & \sffx{(y)AsI} & VJ \\
\mtag[def]{sal}  & \sffx{sAl}    & VJ \\
\mtag[def]{la}   & \sffx{lA}     & NV  OV \\
\bottomrule
\end{tabular}
\end{center}
\end{table}


\section{About ambiguity and overgeneration}

TRmorph produces more ambiguous analyses than the others (mainly based
on \cite{oflazer1994}) reported in the literature. This is partially
because of the 

\section{\label{sec:other-tools}Other tools}

\subsection{Stemming and lemmatization}

\subsection{Unknown word guesser}

\subsection{Hyphenation}

\subsection{Morphological segmentation}

\printbibliography

\printindex

\end{document}
